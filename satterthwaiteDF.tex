\documentclass[12pt]{article}
%\documentclass[useAMS,usenatbib]{biom}
%\documentclass[useAMS,usenatbib,referee]{biom}
\usepackage{epsfig, amsmath, amsfonts, verbatim, setspace, tabularx}
\usepackage[authoryear,round]{natbib}
\usepackage[table]{xcolor}
\usepackage{hyperref}

\title{Computing Satterthwaite degrees of freedom for glmer models}
\author{James P. Hughes\\
Department of Biostatistics \\
University of Washington \\
Seattle, WA 98195, U.S.A. \\
\date{\today}
}
\setlength{\textwidth}{6.5in}
\setlength{\oddsidemargin}{0in}
\setlength{\topmargin}{0in}
\setlength{\textheight}{8in}

\newcommand{\beq}{\begin{equation}}
\newcommand{\eeq}{\end{equation}}
\newcommand{\bdm}{\begin{displaymath}}
\newcommand{\edm}{\end{displaymath}}
\newcommand{\E}{{\bf E}}
\renewcommand{\thempfootnote}{\alph{mpfootnote}}
\renewcommand{\baselinestretch}{1.5}

\newcolumntype{Z}{>{\centering\arraybackslash}X}

\bibliographystyle{apalike}

\begin{document}

%\begin{abstract}

%\end{abstract}
%\begin{keywords}
%Design-based inference; Permutation test; Stepped wedge.
%\end{keywords} 

\maketitle{}

\section{Introduction}
\label{sec:intro}
The following is based on the SAS documentation at \url{https://documentation.sas.com/doc/en/statug/15.2/statug_glimmix_details39.htm}. Also critical was the paper ``Fitting Linear Mixed-Effects Models Using lme4'' by Bates et al (Journal of Statistical Software, 67(1), 1-48, 2015).


\section{Methods}
\label{sec:methods}
\subsection{lmer models}
\subsubsection{The 1 degree of freedom case}
Consider the t-statistic 
\beq
t = \frac{\ell^{\prime} \hat{\beta}}{\sqrt{\ell^{\prime} C \ell}}
\eeq
where i$\ell$ is a $p \times 1$ matrix that defines the contrast of interest, $\hat{\beta}$ is the fixed effects coefficient vector estimate (of length $p$), and $C$ is the variance-covariance matrix of the fixed effects. $C$ depends on $\sigma$ and the random effects parameters $\theta$ (of length $q$)
We can compute the Satterthwaite degrees of freedom for this statistic as
\beq
\nu = \frac{2 (\ell^{\prime} C \ell)^2}{g^{\prime} A g}
\eeq
where $A$ is the variance-covariance matrix of the random effects parameter estimates, $\hat{\theta}$, and $g$ is the first derivative of $\ell^{\prime} C \ell$ with respect to $\theta$, evaluated at $\hat{\theta}$.

Given a lmer fitted object, rslt, we can extract
\begin{eqnarray*}
\hat{\theta} &=& \mbox{getME(rslt,"theta")} \\
\hat{\beta} &=& \mbox{fixef(rslt)} \\
C &=& \mbox{vcov(rslt)} \\
A &=& \mbox{MASS::ginv(rslt@optinfo\$derivs\$Hessian/2)[1:q,1:q]}
\end{eqnarray*}
where $q$ is the number of random effects parameters (not including $\sigma$). Note that this code gives the so-called ``theta'' formulation of the random effects parameters (instead of the variance formulation). The difficulty is getting $g$ (or equivalently, ${d C}/{d \theta}$). to do this we use the grad function (important: use right-sided derivatives) from the numDeriv package by calling 

\begin{verbatim}
g = grad(func_lmer,theta,obj=rslt,lvec=lvec)}
\end{verbatim}

The key function, \verb+func_lmer+, 
returns an updated value of $\ell^{\prime} C \ell$ for a given value of theta. It does this by extracting information from the fitted object (rslt) and recomputing $C$ as a function of $\theta$, following information provided in the Bates et al article (particularly critical was equation 54 and section 3.6). See the code for \verb+func_lmer+.

\subsubsection{The multi-degree of freedom case}
Let $\ell$ be an $r \times p$ matrix defining the $r$ contrasts of interest so that the hypothesis of intertest is
\beq
Ho: L \hat{\beta} = 0
\eeq

Define the $F$ statistic as
\bdm
F = \hat{\beta}^\prime L^\prime (L C L^\prime)^{-1} L \hat{\beta}
\edm
$F$ is distributed as according to an F-distribution with $r$ and $df$ degrees of freedom where $df$ is computed as follows. 

Compute the spectral decomposition (R function eiger) of $L C L^\prime = U^\prime D U$. Define $b_j$ as the $j$'th row of $U L$ (an $r \times p$ matrix). Then compute $g_j$, the gradient of $b_j C b_j^\prime$ with respect to $\theta$ (so $g_j$ is a vector of length $q$). We use the jacobian function from the numDeriv package (important: use right-sided derivatives), along with the \verb+func_lmer+  function to compute $g_j$.

Next, compute
\bdm
\nu_j = \frac{2 (D_j)^2}{g_j^{\prime} A g_j}
\edm
for $j = 1 \ldots r$, and let
\bdm
E = \sum_{j=1}^r \frac{\nu_j}{\nu_j -2} I(\nu_j > 2)
\edm

Then
\bdm
df = \frac{E}{E - r}
\edm
if $E > r$ and $0$ otherwise.

%\section{Simulation Results}
%\label{sec:results}

%\section{Example} 
%\label{sec:example}

\section{Discussion} 
\label{sec:discuss}


\vspace{.5in}

\noindent
Acknowledgements: This research was supported by NIH grant AI29168.

%\clearpage
%\pagebreak
%\bibliography{References}

%\section{Supporting Information}


%\section*{\underline{Appendix}}


\end{document}
